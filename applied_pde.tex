%% Based on a TeXnicCenter-Template by Gyorgy SZEIDL.
%%%%%%%%%%%%%%%%%%%%%%%%%%%%%%%%%%%%%%%%%%%%%%%%%%%%%%%%%%%%%

%----------------------------------------------------------
%
\documentclass{amsbook}%
%
%----------------------------------------------------------
% This is a sample document for the AMS LaTeX Book or Monograph Class
% Class options
%       --  Body text point size:
%                        8pt, 9pt, 10pt (default), 11pt, 12pt
%       --  Paper size:  letterpaper (8.5x11 inch, default), a4paper
%       --  Orientation: portrait(default), landscape
%       --  Print side:  oneside, twoside (default)
%       --  Quality:     final(default), draft
%       --  Title page:  titlepage, notitlepage
%       --  Start chapter on left:
%                        openright (no, default), openany
%       --  Columns:     onecolumn (default), twocolumn
%       --  Omit extra math features:
%                        nomath
%       --  AMS fonts (noamasfonts available):
%                        noamsfonts
%       --  PSAMSfonts (fewer AMSfontsizes)
%                        psamsfonts
%       --  Equation numbering (equation numbers on the left is the default)
%                        leqno (default), reqno
%       --  Equation centering (equations centered is the default)
%                        centeredtags (default}, tbtags (top, bottom)
%       --  Displayed equations (centered is the default)
%                        fleqn (flush left)
% For instance the command
%          \documentclass[a4paper,12p,reqno]{amsbook}
% ensures that the paper size is a4, fonts are typeset at the size 12p
% and the equation numbers are on the right side.
%
\usepackage{amsmath}%
\usepackage{amsfonts}%
\usepackage{amssymb}%
\usepackage{graphicx}
%----------------------------------------------------------
\theoremstyle{plain}
\newtheorem{acknowledgement}{Acknowledgement}
\newtheorem{algorithm}{Algorithm}
\newtheorem{axiom}{Axiom}
\newtheorem{case}{Case}
\newtheorem{claim}{Claim}
\newtheorem{conclusion}{Conclusion}
\newtheorem{condition}{Condition}
\newtheorem{conjecture}{Conjecture}
\newtheorem{corollary}{Corollary}
\newtheorem{criterion}{Criterion}
\newtheorem{definition}{Definition}
\newtheorem{example}{Example}
\newtheorem{exercise}{Exercise}
\newtheorem{lemma}{Lemma}
\newtheorem{notation}{Notation}
\newtheorem{problem}{Problem}
\newtheorem{proposition}{Proposition}
\newtheorem{remark}{Remark}
\newtheorem{solution}{Solution}
\newtheorem{summary}{Summary}
\newtheorem{theorem}{Theorem}
\numberwithin{equation}{section}
%-----------------------------------------------------------
\begin{document}
\frontmatter
\title[Applied PDE Solutions]{Solutions Manual to Applied Partial Differential Equations}
\author[S. Nash]{Stewart Nash}
\address[S. Nash]{362 Lowell Street%
\hspace*{\fill}\linebreak\indent%
Andover, MA 01810}%
%\curraddr[P. Carlson]{Author One, current address, line 1
%\hspace*{\fill}\linebreak
%\indent Author One, current address, line 2}%
\email[S. Nash]{Stewart.M.Nash@gmail.com}
\urladdr{http://www.pervigilent.com}
\thanks{The Author thanks DuChateau and Zachmann for such a good book. The internet for endless resources.}
%\subjclass{Primary 05C38, 15A15; Secondary 05A15, 15A18}
\dedicatory{Dedicated to those who came before me and those who will come after.}

\begin{abstract}
This work consists of solutions to the exercises from the volume ``Applied Partial Differential Equations'' by Paul DuChateau and David Zachmann.
\end{abstract}
\maketitle
\tableofcontents

\chapter*{Preface}

\markboth{PREFACE}{PREFACE} This document consists of computer-based solutions to problems in ``Applied Partial Differential Equations'' by Paul DuChateau and David Zachmann.

\mainmatter

%\part{The First Part}

\chapter{Mathematical Modeling and Partial Differential Equations}

	\section{Equation of Heat Conduction}

	\noindent\textbf{1}. Consider an infinitely long rod for which the parameters $K$, $\epsilon$, $\sigma$, $C$ are such that $\beta = 0.1$. Then equation (1.2.8*) becomes
	\begin{equation}
		u^{j+1}_n=0.1u^j_{n+1}+0.8u^j_n+0.1u^j_{n-1}
	\end{equation}
	Suppose
	\begin{equation}
		u^0_n
		\begin{cases}
			1 & \text{for $n=4,5,6$}\\
			0 & \text{for all other $n$}
		\end{cases}
	\end{equation}
	Then use (1.2.8*) and this initial condition to compose $u^j_n$ for $n=-5,\dots,5$ for $j=1,\dots,5$. For each value of $j$, for how many $n$ is $u^j_n$ different from zero?
	\\[12pt]

	\noindent\textbf{2.} Repeat Exercise 1 for the situation in which the rod is of finite length $L$ with $10\epsilon=L$. Suppose
	\begin{equation}
		u^j_0=1\,\text{and}\quad u^j_{10}=-1\quad\text{for all}\,j>0
	\end{equation}
	and
	\begin{equation}
		u^0_n=0\quad\text{for all}\,n
	\end{equation}
	Then use (1.2.8*) to compute $u^j_n$ for $n=1,\ldots,9$ and $j=1,\ldots,5$.
	\\[12pt]
\begin{small}
\begin{verbatim}
import numpy as np
import matplotlib.pyplot as plt

N = 10 # Length of rod
T = 5 # Duration of simulation

def matrix_power(x, n):
    y = x.copy()
    if n > 1:
        for i in np.arange(n - 1):
            y = np.matmul(y, x)
    return y

coefficients = np.zeros((N, N))
for i in np.arange(N):
    if i == 0:
        coefficients[i, 0]  = 1
    elif i == N - 1:
        coefficients[i, N - 1] = 1
    else:
        coefficients[i, i - 1] = 0.1
        coefficients[i, i] = 0.8
        coefficients[i, i + 1] = 0.1

initial_conditions = np.zeros((N, 1))
initial_conditions[0, 0] = 1
initial_conditions[N - 1, 0] = -1
y = [np.squeeze(np.transpose(np.matmul(matrix_power(coefficients, i),\
	initial_conditions))) for i in np.arange(T)]

length_intervals = np.arange(N)
plt.figure()
for i in np.arange(T):
    plt.plot(length_intervals, y[i])
plt.show()
print("The number of non-zero elements is {0}.".format(\
	np.count_nonzero(y[T - 1] != 0)))
\end{verbatim}
\end{small}

\chapter{Finite Difference Methods for Parabolic Equations}

	\section{Computational Methods}
	
		\noindent\textbf{4.} Use Algorithm 8.1 to approximate the solution of the initial-boundary-value problem
		\begin{align}
			u_t-u_{xx}=-2e^{x-t}, & 0<x<1,\, t>0\\
			u(x,0)=e^x, & 0<x<1\\
			u(0,t)=e^{-t}, u(1,t)=e^{1-t}, & t>0
		\end{align}
		\textbf{(a)} Choose $k=0.0025$ and $nmax=9$ (so $h=0.1$) and compare the numerical and exact solutions, $u(x,t)=e^{x-t}$, at time $t=0.5$.\\
		\textbf{(b)} Choose $k=0.01$ and $nmax=9$ and explain the numerical results.
		\\[12pt]
\begin{small}
\begin{verbatim}
import numpy as np

# Forward Difference Method - Dirichlet Initial-Boundary-Value Problem
def algorithm_8_1(diffusivity,
                  endpoint,
                  time_step,
                  number_of_time_steps,
                  number_of_nodes,
                  right_side,
                  initial_condition,
                  boundary_condition_left,
                  boundary_condition_right):
    # Define a grid
    increment = endpoint / (number_of_nodes + 1)
    coefficient_r = diffusivity * time_step / increment ** 2
    if coefficient_r > 0.5:
        print("WARNING: algorithm_8_1 is unstable")
        
    # Initialize numerical solution
    t = np.zeros((number_of_time_steps + 1,))
    #x = np.zeros((1, number_of_nodes + 2))
    x = np.zeros((number_of_nodes + 2,))
    x[0] = 0
    #V = np.zeros((1, number_of_nodes + 2))
    V = np.zeros((number_of_nodes + 2,))
    V[0] = (boundary_condition_left(0) + initial_condition(0)) / 2
    for n in np.arange(number_of_nodes):
        x[n + 1] = x[n] + increment
        V[n + 1] = initial_condition(x[n + 1])
    x[number_of_nodes + 1] = endpoint
    V[number_of_nodes + 1] = (boundary_condition_right(0) + \
    	initial_condition(endpoint)) / 2

    # Begin time stepping
    #U = np.zeros((1, number_of_nodes + 2))
    U = np.zeros((number_of_nodes + 2,))    
    for j in np.arange(number_of_time_steps):
        # Advance solution one time step
        for n in np.arange(number_of_nodes):
            U[n + 1] = coefficient_r * V[n]
            U[n + 1] += (1 -  2 * coefficient_r) * V[n + 1]
            U[n + 1] += coefficient_r * V[n + 2]
            U[n + 1] += time_step * right_side(x[n + 1], t[j])
        t[j + 1] = t[j]  + time_step
        U[0] = boundary_condition_left(t[j + 1])
        U[number_of_nodes + 1] = boundary_condition_right(t[j + 1])        
        # Output numerical solution
        # Prepare for next time step
        for n in np.arange(number_of_nodes + 2):
            V[n] = U[n]
            
    #x = x[1:-1]
    t = t[:-1]
            
    return U, x, t

import math

# right_side
def S(x, t):
    return -2.0 * math.e ** (x - t)

# initial_condition
def f(x):
    return math.e ** x

# boundary_condition_left
def p(t):
    return  math.e ** -t

# boundary_condition_right
def q(t):
    return math.e ** (1 - t)

# exact_answer	
def u(x, t):
	return math.e ** (x - t)

a2 = 1 # diffusivity
L = 1 # endpoint
k = 0.0025 # time_step
nmax = 9 # number_of_nodes
end_time = 0.5
jmax = int(end_time / k) # number_of_time_steps

numerical_answer, x, t = algorithm_8_1(a2,
                       L,
                       k,
                       jmax,
                       nmax,
                       S,
                       f,
                       p,
                       q)
exact_answer = 	u(x, t[-1])
answer_error = (numerical_answer - exact_answer) / (exact_answer)
answer_error = answer_error * 100

from matplotlib import pyplot as plt

fig, ax1 = plt.subplots()

ax1.set_xlabel('Position')
ax1.set_ylabel('Temperature')
ax1.plot(x, numerical_answer, 'r', label='Numerical Solution')
ax1.plot(x, exact_answer, 'g', label='Exact Solution')

ax2 = ax1.twinx()
ax2.set_ylabel('Percent Error')
ax2.plot(x, answer_error, 'b', label='Percent Error')

ax1.legend()
ax2.legend()

plt.show()
\end{verbatim}
\end{small}
		
		\noindent\textbf{5.} Use Algorithm 8.3 or 8.4 to approximate the solution of the initial-boundary-value problem 
		\begin{align}
			u_t-u_{xx}=-2e^{x-t}, & 0<x<1,\, t>0\\
			u(x,0)=e^x, & 0<x<1\\
			u(0,t)=e^{-t}, u(1,t)=e^{1-t}, & t>0
		\end{align}
		\textbf{(a)} Choose $k=0.0025$ and $nmax=9$ (so $h=0.1$) and compare the numerical and exact solutions, $u(x,t)=e^{x-t}$, at time $t=0.5$.\\
		\textbf{(b)} Choose $k=0.01$ and $nmax=9$ and compare the numerical and exact solutions at time $t=0.5$.\\
		\textbf{(c)} Choose $k=0.01$ and $nmax=99$ (so $h=0.01$) and compare the numerical and exact solutions at time $t=0.5$ at the positions $x=0.1,0.2,\ldots ,0.9$.
		\\[12pt]

\begin{small}
\begin{verbatim}
import numpy as np

# Solution of a Tridiagonal Linear System
def algorithm_8_2(a, # subdiagonal
                 b, # diagonal
                 c, # superdiagonal
                 d, # right-hand side
                 number_of_nodes=None):
    if number_of_nodes is None:
        number_of_nodes = d.size
    # Forward substitute to eliminate subdiagonal
    for n in np.arange(number_of_nodes - 1):
        ratio = a[n + 2] / b[n + 1]
        b[n + 2] = b[n + 2] - ratio * c[n + 1]
        d[n + 2] = d[n + 2] - ratio * d[n + 1]
    # Back substitude and store in solution array in d
    d[number_of_nodes] = d[number_of_nodes] / b[number_of_nodes]
    for l in np.arange(number_of_nodes - 1):
        n = number_of_nodes - (l + 1)
        d[n] = (d[n] - c[n] * d[n + 1]) / b[n]
    return d
        
# Backward Difference Method - Dirichlet Initial-Boundary-Value Problem
def algorithm_8_3(diffusivity,
                  endpoint,
                  time_step,
                  number_of_time_steps,
                  number_of_nodes,
                  right_side,
                  initial_condition,
                  boundary_condition_left,
                  boundary_condition_right):
    # Define a grid
    increment = endpoint / (number_of_nodes + 1)
    coefficient_r = diffusivity * time_step / increment ** 2
        
    # Initialize numerical solution
    t = np.zeros((number_of_time_steps + 1,))
    x = np.zeros((number_of_nodes + 2,))
    x[0] = 0
    U = np.zeros((number_of_nodes + 2,))
    U[0] = (boundary_condition_left(0) + initial_condition(0)) / 2
    for n in np.arange(number_of_nodes):
        x[n + 1] = x[n] + increment
        U[n + 1] = initial_condition(x[n + 1])
    U[number_of_nodes + 1] = (boundary_condition_right(0) + \
    	initial_condition(endpoint)) / 2
    x[number_of_nodes + 1] = endpoint

    term_a = np.zeros((number_of_nodes + 2,))
    term_b = np.zeros((number_of_nodes + 2,))
    term_c = np.zeros((number_of_nodes + 2,))
    term_d = np.zeros((number_of_nodes + 2,))    
    # Begin time stepping  
    for j in np.arange(number_of_time_steps):
        # Define tridiagonal system
        t[j + 1] = t[j]  + time_step
        for n in np.arange(number_of_nodes):
            term_a[n + 1] = - coefficient_r
            term_b[n + 1] = 1 + 2 * coefficient_r
            term_c[n + 1] = - coefficient_r
            term_d[n + 1] = U[n + 1] + time_step * right_side(x[n + 1], t[j + 1])
        term_d[1] = term_d[1] + coefficient_r * boundary_condition_left(t[j + 1])
        term_d[number_of_nodes] = term_d[number_of_nodes] + \ 
        	coefficient_r * boundary_condition_right(t[j + 1])
        # Advance solution one time step
        term_d = algorithm_8_2(term_a, term_b, term_c, term_d, number_of_nodes)
        for n in np.arange(number_of_nodes):
            U[n + 1] = term_d[n + 1]
        U[0] = boundary_condition_left(t[j + 1])        
    # Output numerical solution        
    U[0] = boundary_condition_left(t[j + 1])
    U[number_of_nodes + 1] = boundary_condition_right(t[j + 1])
            
    return U, x, t

import math

# right_side
def S(x, t):
    return -2.0 * math.e ** (x - t)

# initial_condition
def f(x):
    return math.e ** x

# boundary_condition_left
def p(t):
    return  math.e ** -t

# boundary_condition_right
def q(t):
    return math.e ** (1 - t)

# exact_answer	
def u(x, t):
	return math.e ** (x - t)

a2 = 1 # diffusivity
L = 1 # endpoint
k = 0.0025 # time_step
nmax = 9 # number_of_nodes
end_time = 0.5
jmax = int(end_time / k) # number_of_time_steps

numerical_answer, x, t = algorithm_8_3(a2,
                       L,
                       k,
                       jmax,
                       nmax,
                       S,
                       f,
                       p,
                       q)
exact_answer = u(x, t[-1])
answer_error = (numerical_answer - exact_answer) / (exact_answer)
answer_error = answer_error * 100

from matplotlib import pyplot as plt

fig, ax1 = plt.subplots()

ax1.set_xlabel('Position')
ax1.set_ylabel('Temperature')
ax1.plot(x, numerical_answer, 'r', label='Numerical Solution')
ax1.plot(x, exact_answer, 'g', label='Exact Solution')

ax2 = ax1.twinx()
ax2.set_ylabel('Percent Error')
ax2.plot(x, answer_error, 'b', label='Percent Error')

ax1.legend()
ax2.legend()

plt.show()
\end{verbatim}
\end{small}
	
		\noindent\textbf{6.} Use Algorithm 8.5 to approximate the solution of the initial-boundary-value problem 
		\begin{align}
			u_t-u_{xx}=-2e^{x-t}, & 0<x<1,\, t>0\\
			u(x,0)=e^x, & 0<x<1\\
			u(0,t)=e^{-t}, u(1,t)=e^{1-t}, & t>0
		\end{align}
		\textbf{(a)} Choose $k=0.0025$ and $nmax=9$ (so $h=0.1$) and compare the numerical and exact solutions, $u(x,t)=e^{x-t}$, at time $t=0.5$.\\
		\textbf{(b)} Choose $k=0.01$ and $nmax=9$ and compare the numerical and exact solutions at time $t=0.5$.\\
		\textbf{(c)} Choose $k=0.01$ and $nmax=99$ (so $h=0.01$) and compare the numerical and exact solutions at time $t=0.5$.
		\\[12pt]
		

\begin{small}
\begin{verbatim}
import numpy as np
        
# Backward Difference Method - Neumann Initial-Boundary-Value Problem
def algorithm_8_5(diffusivity,
                  endpoint,
                  time_step,
                  number_of_time_steps,
                  number_of_nodes,
                  right_side,
                  initial_condition,
                  boundary_condition_left,
                  boundary_condition_right):
    # Define a grid
    increment = endpoint / (number_of_nodes - 1)
    coefficient_r = diffusivity * time_step / increment ** 2
        
    # Initialize numerical solution
    t = np.zeros((number_of_time_steps + 1,))
    x = np.zeros((number_of_nodes + 1,))
    x[0] = -increment
    U = np.zeros((number_of_nodes + 1,)) 
    for n in np.arange(number_of_nodes):
        x[n + 1] = x[n] + increment
        U[n + 1] = initial_condition(x[n + 1])

    term_a = np.zeros((number_of_nodes + 1,))
    term_b = np.zeros((number_of_nodes + 1,))
    term_c = np.zeros((number_of_nodes + 1,))
    term_d = np.zeros((number_of_nodes + 1,))    
    # Begin time stepping  
    for j in np.arange(number_of_time_steps):
        # Define tridiagonal system
        t[j + 1] = t[j]  + time_step
        for n in np.arange(number_of_nodes):
            term_a[n + 1] = - coefficient_r
            term_b[n + 1] = 1 + 2 * coefficient_r
            term_c[n + 1] = - coefficient_r
            term_d[n + 1] = U[n + 1] + time_step * right_side(x[n + 1], t[j + 1])
        term_d[1] = term_d[1] - \
        	2 * increment * coefficient_r * boundary_condition_left(t[j + 1])
        term_d[number_of_nodes] = term_d[number_of_nodes] + \
        	2 * increment * coefficient_r * boundary_condition_right(t[j + 1])
        term_c[1] = -2 * coefficient_r
        term_a[number_of_nodes] = -2 * coefficient_r
        # Advance solution one time step
        term_d = algorithm_8_2(term_a, term_b, term_c, term_d, number_of_nodes)
        for n in np.arange(number_of_nodes):
            U[n + 1] = term_d[n + 1]   
    # Output numerical solution

    #t = t[:-1]
    x = x[1:]
    U = U[1:]
            
    return U, x, t

import math

# right_side
def S(x, t):
    return -2.0 * math.e ** (x - t)

# initial_condition
def f(x):
    return math.e ** x

# boundary_condition_left
def p(t):
    return  math.e ** -t

# boundary_condition_right
def q(t):
    return math.e ** (1 - t)

# exact_answer	
def u(x, t):
	return math.e ** (x - t)

a2 = 1 # diffusivity
L = 1 # endpoint
k = 0.0025 # time_step
nmax = 9 # number_of_nodes
end_time = 0.5
jmax = int(end_time / k) # number_of_time_steps

numerical_answer, x, t = algorithm_8_5(a2,
                       L,
                       k,
                       jmax,
                       nmax,
                       S,
                       f,
                       p,
                       q)
exact_answer = u(x, t[-1])
answer_error = (numerical_answer - exact_answer) / (exact_answer)
answer_error = answer_error * 100

from matplotlib import pyplot as plt

fig, ax1 = plt.subplots()

ax1.set_xlabel('Position')
ax1.set_ylabel('Temperature')
ax1.plot(x, numerical_answer, 'r', label='Numerical Solution')
ax1.plot(x, exact_answer, 'g', label='Exact Solution')

ax2 = ax1.twinx()
ax2.set_ylabel('Percent Error')
ax2.plot(x, answer_error, 'b', label='Percent Error')

ax1.legend()
ax2.legend()

plt.show()
\end{verbatim}
\end{small}

	\section{Fourier's Method for Difference Equations}
	
		\noindent\textbf{5.} Given the initial-boundary-value problem
		\begin{align}
			u_t-u_{xx}=0, & 0<x<1,\, t>0\\
			u(x,0)=x^2, & 0<x<1\\
			u_x(0,t)=0=u_x(1,t), & t>0
		\end{align}
		and the grid $x_1=0$, $x_2=1/2$, and $x_3=1$, so $N=3$,\\
		\textbf{(a)} Use the matrix $\mathbf{F}_2$ to formulate a forward-in-time difference system.\\
		\textbf{(b)} Use the matrix $\mathbf{F}_2$ to formulate a backward-in-time difference system.\\
		\textbf{(c)} Use the matrix $\mathbf{F}_2$ to formulate a Crank-Nicolson difference system.\\
		\textbf{(d)} Use Fourier's method to solve the system (a).\\
		\textbf{(e)} Use Fourier's method to solve the system (b).\\
		\textbf{(f)} Use Fourier's method to solve the system (c).\\[12pt]

\begin{small}
\begin{verbatim}
import numpy as np
import math

def node_points(endpoint, number_of_nodes):
    increment = endpoint / (number_of_nodes - 1)
    x = [i * increment for i in range(number_of_nodes)]
    #x = np.array(x)
    return x

def eigenvalue(n, N):
    return 2 * (1 - math.cos((n - 1) * math.pi / (N - 1)))

# eigenvector Vn
def eigenvector_v(n, N, L):
    #x = [((i + 1) - 1) / (N - 1) for i in range(N)]
    x = node_points(L, N) 
    output = [math.cos((n - 1) * math.pi * y) for y in x]
    output = np.array(output)
    return output

# eigenvector Wn
def eigenvector_w(n, N, L):
    #x = [((i + 1) - 1) / (N - 1) for i in range(N)]
    x = node_points(L, N)
    output = [2 * math.cos((n - 1) * math.pi * y) for y in x]
    output[0] = output[0] / 2
    output[-1] = output[-1] / 2
    output = np.array(output)
    return output

# initial_condition
def f(x):
    return x ** 2

def coefficients(N, initial_condition, L):
    output = [coefficient(i + 1, N, initial_condition, L) for i in range(N)]
    output = np.array(output)
    return output
    
def coefficient(n, N, initial_condition, L):
    x = node_points(L, N)
    initial = [initial_condition(y) for y in x]
    initial = np.array(initial)
    output = np.dot(initial, eigenvector_w(n, N, L))
    output = output / np.dot(eigenvector_v(n, N, L), eigenvector_w(n, N, L))
    return output

# Forward-in-time (FIT)
# Neumann-Neumann (NN) Initial-Boundary-Value Problem
def solve_nn_fit(endpoint,
                  time_step,
                  number_of_time_steps,
                  number_of_nodes,
                  right_side,
                  initial_condition,
                  boundary_condition_left,
                  boundary_condition_right):
    j = number_of_time_steps
    increment = endpoint / (number_of_nodes - 1)
    coefficient_r = time_step / increment ** 2
    matrix_b = lambda number: 1 - coefficient_r * number
    matrix_a = lambda number: 1
    U = np.zeros((number_of_nodes,)) 
    for n in np.arange(number_of_nodes):
        l = eigenvalue(n + 1, number_of_nodes)
        V = eigenvector_v(n + 1, number_of_nodes, endpoint)
        c_n = coefficient(n + 1, number_of_nodes, initial_condition, endpoint)
        U = U + (matrix_b(l) / matrix_a(l)) ** j * c_n * V   
    return U

# Backward-in-time (BIT)
# Neumann-Neumann (NN) Initial-Boundary-Value Problem
def solve_nn_bit(endpoint,
                  time_step,
                  number_of_time_steps,
                  number_of_nodes,
                  right_side,
                  initial_condition,
                  boundary_condition_left,
                  boundary_condition_right):
    U = np.zeros((number_of_nodes,))
    return U

# Crank-Nicolson (CN)
# Neumann-Neumann (NN) Initial-Boundary-Value Problem
def solve_nn_cn(endpoint,
                  time_step,
                  number_of_time_steps,
                  number_of_nodes,
                  right_side,
                  initial_condition,
                  boundary_condition_left,
                  boundary_condition_right):
    U = np.zeros((number_of_nodes,))
    return U

L = 1 # endpoint
k = 0.0025 # time_step
nmax = 3 # number_of_nodes
end_time = 0.5
jmax = int(end_time / k) # number_of_time_steps


answer_a = solve_nn_fit(L,
                  k,
                  jmax,
                  nmax,
                  0,
                  f,
                  0,
                  0)
\end{verbatim}
\end{small}
		
		\noindent\textbf{(a)}
		\begin{align}
			A(\mathbf{F}_2)\mathbf{U}^{j+1} & = B(\mathbf{F}_2)\mathbf{U}^j\\
			\mathbf{U}^0 & = \mathbf{f}\\
			c^0_n & = \frac{\mathbf{f}\cdot\mathbf{W}_n}{\mathbf{V}_n\cdot\mathbf{W}_n}\\
			A(\lambda) & = 1\\
			B(\lambda) & = 1-r\lambda
		\end{align}
		\textbf{(b)}
		\begin{align}
			A(\mathbf{F}_2)\mathbf{U}^{j+1} & = B(\mathbf{F}_2)\mathbf{U}^j\\
			\mathbf{U}^0 & = \mathbf{f}\\
			c^0_n & = \frac{\mathbf{f}\cdot\mathbf{W}_n}{\mathbf{V}_n\cdot\mathbf{W}_n}\\
			A(\lambda) & = 1+r\lambda\\
			B(\lambda) & = 1
		\end{align}
		\textbf{(c)}
		\begin{align}
			A(\mathbf{F}_2)\mathbf{U}^{j+1} & = B(\mathbf{F}_2)\mathbf{U}^j\\
			\mathbf{U}^0 & = \mathbf{f}\\
			c^0_n & = \frac{\mathbf{f}\cdot\mathbf{W}_n}{\mathbf{V}_n\cdot\mathbf{W}_n}\\
			A(\lambda) & = 1+r\lambda/2\\
			B(\lambda) & = 1-r\lambda/2
		\end{align}
				
		\noindent\textbf{6.} Given the initial-boundary-value problem
		\begin{align}
			u_t-u_{xx}=0, & 0<x<1,\, t>0\\
			u(x,0)=x^2, & 0<x<1\\
			u_x(0,t)=0=u_x(1,t), & t>0
		\end{align}
		and the grid $x_1=0$, $x_2=1/3$, and $x_3=2/3$, so $N=3$,\\
		\textbf{(a)} Use the matrix $\mathbf{F}_2$ to formulate a forward-in-time difference system.\\
		\textbf{(b)} Use the matrix $\mathbf{F}_2$ to formulate a backward-in-time difference system.\\
		\textbf{(c)} Use the matrix $\mathbf{F}_2$ to formulate a Crank-Nicolson difference system.\\
		\textbf{(d)} Use Fourier's method to solve the system (a).\\
		\textbf{(e)} Use Fourier's method to solve the system (b).\\
		\textbf{(f)} Use Fourier's method to solve the system (c).\\[12pt]
		
		\noindent\textbf{(a)}
		\begin{align}
			A(\mathbf{F}_3)\mathbf{U}^{j+1} & = B(\mathbf{F}_3)\mathbf{U}^j\\
			\mathbf{U}^0 & = \mathbf{f}\\
			A(\lambda) & = 1\\
			B(\lambda) & = 1-r\lambda\\			
			c^j_n & = \frac{\mathbf{U}^j\cdot\mathbf{W}_n}{\mathbf{V}_n\cdot\mathbf{W}_n}\\
			& = \left[\frac{B(\lambda_n)}{A(\lambda_n)}\right]^jc^0_n\\
			& = \left[\frac{B(\lambda_n)}{A(\lambda_n)}\right]^j\frac{\mathbf{f}\cdot\mathbf{W}_n}{\mathbf{V}_n\cdot\mathbf{W}_n}\\
			\mathbf{U}^j & = \sum_{n=0}^N{c^j_n\mathbf{V}_n}\\
			& = \sum_{n=0}^N{\left[\frac{B(\lambda_n)}{A(\lambda_n)}\right]^jc^0_n\mathbf{V}_n}\\
			& = \sum_{n=0}^N{(1-r\lambda_n)^{-j}c^0_n\mathbf{V}_n}\\
		\end{align}
		\textbf{(b)}
		\begin{align}
			A(\mathbf{F}_3)\mathbf{U}^{j+1} & = B(\mathbf{F}_3)\mathbf{U}^j\\
			\mathbf{U}^0 & = \mathbf{f}\\
			c^0_n & = \frac{\mathbf{f}\cdot\mathbf{W}_n}{\mathbf{V}_n\cdot\mathbf{W}_n}\\
			A(\lambda) & = 1+r\lambda\\
			B(\lambda) & = 1
		\end{align}
		\textbf{(c)}
		\begin{align}
			A(\mathbf{F}_3)\mathbf{U}^{j+1} & = B(\mathbf{F}_3)\mathbf{U}^j\\
			\mathbf{U}^0 & = \mathbf{f}\\
			c^0_n & = \frac{\mathbf{f}\cdot\mathbf{W}_n}{\mathbf{V}_n\cdot\mathbf{W}_n}\\
			A(\lambda) & = 1+r\lambda/2\\
			B(\lambda) & = 1-r\lambda/2
		\end{align}		
		
		\noindent\textbf{7.} Given the initial-boundary-value problem
		\begin{align}
			u_t-u_{xx}=0, & 0<x<1,\, t>0\\
			u(x,0)=x, & 0<x<1\\
			u_x(0,t)=0=u_x(1,t), & t>0
		\end{align}
		and the grid $x_1=0$, $x_2=1/4, x_3=1/2, x_4=3/4$, and $x_5=1$, so $N=5$,\\
		\textbf{(a)} Use the matrix $\mathbf{F}_2$ to formulate a forward-in-time difference system.\\
		\textbf{(b)} Use the matrix $\mathbf{F}_2$ to formulate a backward-in-time difference system.\\
		\textbf{(c)} Use the matrix $\mathbf{F}_2$ to formulate a Crank-Nicolson difference system.\\
		\textbf{(d)} Use Fourier's method to solve the system (a).\\
		\textbf{(e)} Use Fourier's method to solve the system (b).\\
		\textbf{(f)} Use Fourier's method to solve the system (c).\\[12pt]		
		
\chapter{Numerical Solutions of Hyperbolic Equations}

	\section{Difference Methods for a Scalar Initial-Value Problem}

		\noindent\textbf{1.} Modify Algorithm 9.1 to implement\\
		\textbf{(a)} FTBS method\\
		\textbf{(b)} FTFS method\\
		\textbf{(c)} Lax-Friedrichs method\\
		\textbf{(d)} leapfrog method\\[12pt]
		
		\noindent\textbf{2.} Approximate the solution of the initial-value problem of Example 9.1.3 on the interval $0\leq x\leq 1$ for $0\leq t_j\leq 1.5$ with $h=0.1$ and $k=0.075$ using\\
		\textbf{(a)} FTBS method\\
		\textbf{(b)} Lax-Friederichs method\\
		\textbf{(c)} leapfrog method\\[12pt]
		
		\noindent\textbf{3.} Repeat exercise 2 with $h=0.1$ and $k=0.1$.\\[12pt]
		
		\noindent\textbf{4.} Given the initial-value problem 
		\begin{align}
			u_x+u_t=2xt+x^2, & -\infty<x<\infty,\, t>0\\
			u(x,0)=0, & -\infty<x<\infty\\
		\end{align}
		whose exact solution is $u(x,t)=x^2t$. Find the local truncation error for this problem when using
		\textbf{(a)} FTBS method\\
		\textbf{(b)} Lax-Wendroff method\\[12pt]
		
		\noindent\textbf{8.} Consider the initial-value problem
		\begin{align}
			u_x+u_t&=0, & -\infty<x<\infty,\, t>0\\
			u(x,0)&=
			\begin{cases}
				1, & |x|<0.5,\\
				0, & |x|\geq 0.5,
			\end{cases} & -\infty<x<\infty
		\end{align}		
			

\backmatter

\begin{thebibliography}{9}
\bibitem {DuChateau}DuChateau, P., Zachmann, D. \textit{Applied Partial Differential Equations}, Dover Publications Inc,
Mineola, New York, 1989

\end{thebibliography}

\end{document}
